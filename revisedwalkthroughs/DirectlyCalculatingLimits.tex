\documentclass[../revisedMain.tex]{subfiles}
\begin{document}
\paragraph{Example 1} \[\lim\limits_{q\to 5} \frac{q^2}{q+1}\] To find the limit, we must find the limit from the left-hand and right-hand side. For the left-hand side: \[\lim\limits_{q\to 5^-} \frac{q^2}{q+1}=\frac{25}{6}\] For the right-hand side: 
\[\lim\limits_{q\to 5^+} \frac{q^2}{q+1}=\frac{25}{6}\] The two limits are equal, therefore:
\begin{enumerate}\item The limit \(\lim\limits_{q\to 5} \displaystyle\frac{q^2}{q+1}\) must exist, and\item The limit must have the value \(\displaystyle\frac{25}{6}\).\end{enumerate}
\paragraph{Example 2} \[\lim\limits_{x\to 3}\frac{x^2-9}{x^4-81}\] Again, find the limit from the left-hand and right-hand sides: \[\lim\limits_{x\to 3^-}\frac{x^2-9}{x^4-81}=\frac{0}{0}\] \(\displaystyle\frac{0}{0}\) is a bad number. There is no defined value for \(\displaystyle\frac{0}{0}\), so we should try and work around it to see if we can get an actual answer: 
\begin{equation}\begin{split}&\frac{x^2-9}{x^4-81}=\frac{x^2-9}{(x^2-9)(x^2+9)}\\\implies&\lim\limits_{x\to 3^-}\frac{x^2-9}{x^4-81}=\lim_{x\to 3^-}\frac{1}{x^2+9}=\frac{1}{18}\end{split}\end{equation}
Similarly for \(\lim\limits_{x\to 3^+}\), we run into the same issue and must reduce:
\begin{equation}\begin{split}&\lim\limits_{x\to 3^+}\frac{x^2-9}{x^4-81}=\frac{0}{0}\\\implies&\frac{x^2-9}{x^4-81}=\frac{x^2-9}{(x^2-9)(x^2+9)}\\\implies&\lim\limits_{x\to 3^+}\frac{x^2-9}{x^4-81}=\lim_{x\to 3^+}\frac{1}{x^2+9}=\frac{1}{18}\end{split}
\end{equation} Therefore, the limit must exist and its value is \(\displaystyle\frac{1}{18}\).
\paragraph{Example 3} \[\lim_{y\to\infty}\sin\left(\frac{1}{y}\right)\] We only need to calculate the left-handed limit because we can only approach \(\infty\) from the left side 
\begin{equation}\begin{split}&\lim_{y\to\infty}\sin\left(\frac{1}{y}\right)\\=&\sin\left(\lim_{y\to\infty}\frac{1}{y}\right)\\=&\sin(0)\\=&0\end{split}\end{equation}
\paragraph{Example 4} \[\lim_{t\to\infty}\cos\left(\frac{e^t}{2t+4}\right)\] This has no solution because, unlike Example 3: \[\lim_{t\to\infty}\frac{e^t}{2t+4}\] does not exist and neither can the limit with cosine. This problem does not have a solution.\\
\end{document}