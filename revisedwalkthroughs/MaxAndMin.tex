\documentclass[../revisedmain.tex]{subfiles}
\begin{document}
	\paragraph{Example 1:}Find the relative maxima and minima of the function \\\(f(x)=x^4-x^3-8x^2+12x+3\). Since we have no bounds, we will evaluate this over $(-\infty,\infty)$
	\begin{equation}
		\begin{split}
		f(x)&=x^4-x^3-8x^2+12x+3\\
		f'(x)&=4x^3-3x^2-16x+12\\
		f'(x)&=(4x-3)(x^2)-(4x-3)(4)\\
		f'(x)&=(4x-3)(x^2-4)\\
		f'(x)&=(4x-3)(x-2)(x+2)\\		
		f'(x)&\text{ must be zero at }x=\frac{3}{4},\pm 2\\
		\end{split}
	\end{equation}
	\begin{center}
		Writing out our first derivative test table, we get:\\\vspace{0.25in}
	\begin{tabular}{ |c|c|c|c|c|c|c| } 
		\hline
		(-$\infty$, -2) &-2 & (-2, 0.75) &0.75 &(0.75,2)&2&(2, $\infty$) \\ \hline
		-&0&+&0&-&0&+ \\
		\hline
	\end{tabular}\\\vspace{.25in}
	Because when $x=\pm2$ the derivative changes signs from negative to positive, we know that these two points are relative minimums. Because at $x=0.75$ the derivative changes signs from positive to negative, we know that it is a local maximum. Sure enough, here is the graph for $f(x)$:
	\begin{tikzpicture}
		\begin{axis}[domain=-4:4,axis lines=middle]
		\addplot[domain=-4:4,color=purple,samples=40]{x^4-x^3-8*x^2+12*x+3};
		\addplot[mark=*,color=purple] coordinates{(-2,-29)};
		\addplot[mark=*, color=purple] coordinates{(0.75,7.4)};
		\addplot[mark=*, color=purple] coordinates{(2,3)};
		\end{axis}
	\end{tikzpicture}
	\end{center}
	\paragraph{Example 2:}Find the absolute maximum and minimum of the function\\\(f(x)=\sin(\pi x)\) over the interval [0,0.75]
	\begin{equation}
		\begin{split}
		f(x)&=\sin(\pi x)\\
		f'(x)&=\pi\cos(\pi x)\\
		0&=\pi\cos(\pi x)\quad \{0\le x\le 0.75\}\\
		x&=0.5\\
		\end{split}
	\end{equation}
	\begin{center}
		\begin{tabular}{|c|c|c|}
		\hline
		[0,0.5)&0.5&(0.5,0.75]\\\hline
		+&0&-\\\hline
	\end{tabular}
	\end{center}\vspace{.25in}
	So $x=0.5$ must be an absolute maximum. However, we are also constrained to an interval, so we must check the bounds as well:\[f(0)=0\]\[f(0.5)=1\]\[f(0.75)=\displaystyle\frac{\sqrt{2}}{2}\]From this, we can see that $x=0.5$ is indeed our absolute maximum but $x=0$ is our absolute minimum on this interval as well.\\\\
\end{document}