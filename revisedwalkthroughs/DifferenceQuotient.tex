\documentclass[../revisedmain.tex]{subfiles}
\begin{document}
	\paragraph {Example} For $f(x)=x^2$, the derivative is:
	\begin{equation}
	\begin{split}
	&= \lim_{h\to 0} \frac{(x+h)^2-x^2}{h} \\
	&= \lim_{h\to 0} \frac{x^2+2xh+h^2-x^2}{h} \\
	&= \lim_{h\to 0} \frac{2xh+h^2}{h} \\
	&= \lim_{h\to 0} 2x+h \\
	&= 2x+0 \\
	&= 2x
	\end{split}
	\end{equation}
	So for any point $x$ on the graph of $f(x)=x^2$, the slope of the tangent line will be $y=2x$. Be familiar with the limit of the difference quotient as the definition of the derivative. You will prove more of these in the exercises but the math required is pre-Calculus\footnote{All math before Calculus will be referred to as pre-Calculus in this textbook} and generally unneccesary in the understanding of Calculus.\\
\end{document}