\documentclass[../revisedmain.tex]{subfiles}
\begin{document}
	\newpage
	\paragraph{Example 1:} We already know how to find the slope of a linear equation. We can prove the difference quotient gives the slope for any linear equation. The general slope-intercept form for linear equations is \(f(x)=mx+b\) where \(m\) is the slope and \(b\) is the y-intercept. 
	\begin{equation}
		\begin{split}
		m&\stackrel{?}{=} \lim_{h\to 0}\frac{(m(x+h)+b)-(mx+b)}{h}\\
		m&\stackrel{?}{=} \lim_{h\to 0}\frac{mx+mh+b-mx-b}{h}\\
		m&\stackrel{?}{=} \lim_{h\to 0}\frac{mh}{h}\\
		m&\stackrel{?}{=} \lim_{h\to 0}m\\
		m&=m\\
		&\checkmark\\
	\end{split}
	\end{equation}
	\paragraph{Example 2:} For $f(x)=x^2$, the derivative is:
	\begin{equation}
	\begin{split}
	&= \lim_{h\to 0} \frac{(x+h)^2-x^2}{h} \\
	&= \lim_{h\to 0} \frac{x^2+2xh+h^2-x^2}{h} \\
	&= \lim_{h\to 0} \frac{2xh+h^2}{h} \\
	&= \lim_{h\to 0} 2x+h \\
	&= 2x+0 \\
	&= 2x
	\end{split}
	\end{equation}
	So for any point $x$ on the graph of $f(x)=x^2$, the slope of the tangent line will be $\frac{\Delta y}{\Delta x}=2x$. Be familiar with the limit of the difference quotient as the definition of the derivative. You will prove more of these in the exercises but the math required is pre-Calculus\footnote{All math before Calculus will be referred to as pre-Calculus in this textbook} and generally unneccesary in the understanding of Calculus.\\
	\paragraph{Example 3:} Working off of the previous problem, we can calculate the slope of the tangent line at any point on the graph. For example, the slope at the point (3.5,12.25) is:
	\begin{equation}
		\begin{split}
		\frac{\Delta y}{\Delta x}&=2x\\
		\frac{\Delta y}{\Delta x}&=2(3.5)\\
		\frac{\Delta y}{\Delta x}&=7\\
		\end{split}
	\end{equation}
	Knowing the slope, we can calculate the equation itself. In Calculus, it is often the easiest when finding a tangent line equation to use the point-slope form of a line which is: \[(y-y_0)=m(x-x_0)\] where \(m\) is the slope at a point \((x_0,y_0)\). The tangent line to the graph of \(f(x)=x^2\) at the point (3.5,12.5) is therefore \[y-12.25=7(x-3.5)\] or \[y=7(x-3.5)+12.25\] See the Tips about the AP Test section for more information.\\\\
\end{document}