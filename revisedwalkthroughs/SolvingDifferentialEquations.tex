\documentclass[../revisedmain.tex]{subfiles}
\begin{document}
\paragraph{Example 1:} Solve the differential equation $dy=\frac{-x}{y}dx$ which passes through the point $(0,2)$
\begin{gather*}
	y\,dy=-x\,dx\\
	\int y\,dy=-\int x\,dx\\
	.5y^2+D=-x^2+E\\
\end{gather*}
We don't know either the value of $D$ or $E$, so we can just subtact $D$ from both sides. The benefit of this is we only have one constant now, $E-D$ on the right side. Because $E-D$ is unknown, we can just write it as another constant $.5*C$. We only do this to make the problem a little simpler.
\begin{gather*}
	.5y^2=-.5x^2+.5C\\
	y^2=-x^2+C\\
	y^2+x^2=C\\
	2^2+0^2=C\\
	4=C\\
\end{gather*}
So our final answer is a circle centered at the origin that has a radius of 2: $y^2+x^2=4$.\\
\end{document}