\documentclass[../revisedmain.tex]{subfiles}
\begin{document}
Okay, let's slow down a bit. Here's what we just found out:
\begin{enumerate}
	\item The derivative of an indefinite integral of $f(x)$ is $f(x)$ itself
	\item The integral from $a$ to $b$ of $f(x)$ is $f(b)-f(a)$
\end{enumerate}
This shouldn't be \textit{too} surprising. There's also something else we learned:
\begin{displayquote}
	We can now calculate the (change in) area under a curve without needing to know the bounds!
\end{displayquote}
As you will see later, this is very handy. Note that we can't actually find the exact value of the integral given only the derivative because of the $+C$.\\
This is really cool! The limit of the difference quotient of an \textit{infinite} sum of a function is the function itself! It may seem obvious knowing the rules of derivatives and integrals, but think about it this way:
\[
\lim_{h\to 0}\frac{
	\lim_{n\to\infty}\frac{b-a}{n}\sum_{i=1}^{n} f\left(a+i\frac{b-a}{n}+h\right) - \lim_{n\to\infty}\frac{b-a}{n}\sum_{i=1}^{n} f\left(a+i\frac{b-a}{n}\right) 
}{
h
} = f(x)
\]
How cool is that?! Really cool. Math rocks.\\
\end{document}