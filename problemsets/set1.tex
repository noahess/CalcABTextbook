% Exam Template for UMTYMP and Math Department courses
%
% Using Philip Hirschhorn's exam.cls: http://www-math.mit.edu/~psh/#ExamCls
%
% run pdflatex on a finished exam at least three times to do the grading table on front page.
%
%%%%%%%%%%%%%%%%%%%%%%%%%%%%%%%%%%%%%%%%%%%%%%%%%%%%%%%%%%%%%%%%%%%%%%%%%%%%%%%%%%%%%%%%%%%%%%

% These lines can probably stay unchanged, although you can remove the last
% two packages if you're not making pictures with tikz.
\documentclass[11pt]{exam}
\RequirePackage{amssymb, amsfonts, amsmath, latexsym, verbatim, xspace, setspace}
\RequirePackage{tikz, pgflibraryplotmarks}

% By default LaTeX uses large margins.  This doesn't work well on exams; problems
% end up in the "middle" of the page, reducing the amount of space for students
% to work on them.
\usepackage[margin=1in]{geometry}


% Here's where you edit the Class, Exam, Date, etc.
\newcommand{\class}{AP Calculus AB}
\newcommand{\term}{2016-2017}
\newcommand{\examnum}{Problem Set 1}
\newcommand{\examdate}{3/28/12}
\newcommand{\timelimit}{40 Minutes (Suggested)}

% For an exam, single spacing is most appropriate
\singlespacing
% \onehalfspacing
% \doublespacing

% For an exam, we generally want to turn off paragraph indentation
\parindent 0ex

\begin{document} 
	
	% These commands set up the running header on the top of the exam pages
	\pagestyle{head}
	\firstpageheader{}{}{}
	\runningheader{\class}{\examnum\ - Page \thepage\ of \numpages}{\examdate}
	\runningheadrule
	
	\begin{flushright}
		\begin{tabular}{p{2.8in} r l}
			\textbf{\class} & \textbf{Name (Print):} & \makebox[2in]{\hrulefill}\\
			\textbf{\term} &&\\
			\textbf{\examnum} &&\\
			%\textbf{\examdate} &&\\
			\textbf{Time Limit: \timelimit}
		\end{tabular}\\
	\end{flushright}
	\rule[1ex]{\textwidth}{.1pt}
	
	This set contains \numpages\ pages (including this cover) and \numquestions\ problems. These problems correspond to the topics taught in \textsection 1: Limits. The problems included here are similar to those that will be asked on the AP Test. All answers should be completed to the best of your ability with \textit{all} work shown.
	\newpage % End of cover page
	
	%%%%%%%%%%%%%%%%%%%%%%%%%%%%%%%%%%%%%%%%%%%%%%%%%%%%%%%%%%%%%%%%%%%%%%%%%%%%%%%%%%%%%
	%
	% See http://www-math.mit.edu/~psh/#ExamCls for full documentation, but the questions
	% below give an idea of how to write questions [with parts] and have the points
	% tracked automatically on the cover page.
	%
	%
	%%%%%%%%%%%%%%%%%%%%%%%%%%%%%%%%%%%%%%%%%%%%%%%%%%%%%%%%%%%%%%%%%%%%%%%%%%%%%%%%%%%%%
	
	\begin{questions}
		
		% Basic question
		\addpoints
		\question[5] What condition must be met for a limit to exist at a point $x$ for a function $f(x)$?
		
		% Question with parts
		\newpage
		\addpoints
		\question Consider the function
		\[
		f(x)=
		\begin{cases}
		x & x\leq -1\\
		x^2-1 & -1<x\leq 1\\
		x & x>1 
		\end{cases}
		\]
		\begin{parts}
			\part[5] Find $\lim\limits_{x\to 0} f(x)$.
			\vfill
			\part[5] Find $\lim\limits_{x\to\-1^+}$.
			\vfill
		\end{parts}
		
		% If you want the total number of points for a question displayed at the top,
		% as well as the number of points for each part, then you must turn off the point-counter
		% or they will be double counted.
		\newpage
		\addpoints
		\question[15] Find $\lim\limits_{x\to -1} \displaystyle\frac{\sin\left(\displaystyle\frac{1}{x-1}\right)e^x}{x}$
		
		\newpage
		\addpoints
		\question[10] Consider the function $f(x)=\displaystyle\frac{e^x}{x^{100}}$.
		\noaddpoints % If you remove this line, the grading table will show 20 points for this problem.
		\begin{parts}
			\part[5] Find $\lim\limits_{x\to\infty} f(x)$.
			\vspace{4.5in}
			\part[5] Find $\lim\limits_{x\to\ -\infty} f(x)$.
		\end{parts}
		
		
		
	\end{questions}
\end{document}