\documentclass[../revisedMain.tex]{subfiles}
\begin{document}
	In some cases, it does not make sense to find the derivative with respect to one specific variable in the equation. Perhaps there is another equation that may go with the one you are working with and has different variables. There may also be an advantage if you take  the derivative with respect to some parameter, usually $t$. I like to think that $t$ is sort of like time; you can increase it or decrease it and see how $x$ and $y$ change with respect to this parameter. It is quite simple: all one has to do is take the derivative with respect to this parameter: $\displaystyle\frac{d}{dt} $. For example: 
	\begin{equation}
	\begin{split}
	\frac{d}{dt}( y&=x^2-5x+1)\\
	\frac{dy}{dt}&=2x\frac{dx}{dt}-5\frac{dx}{dt}\\
	\frac{dy}{dt}&=(2x-5)\frac{dx}{dt}
	\end{split}
	\end{equation}
	Note:  while the derivative with respect to $x$ normally ends up being $\displaystyle\frac{dx}{dx}=1$, this is not the case with implicit differentiation.\\
\end{document}