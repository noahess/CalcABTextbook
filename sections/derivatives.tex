\documentclass[../main.tex]{subfiles}
\begin{document}
\subsection{Introduction}
\paragraph{Slopes} It is useful and easy to find the rate of change for any point on a line. It is difficult to approximate how a curved function changes, however. Functions that aren't lines don't have defined slopes and we have to use a tangent line to find the slope at a point. It is difficult to find the tangent line at a point, so we must resort to finding secant approximations for the tangent line.
\begin{center}
\begin{tikzpicture}
	\begin{axis}
	\addplot[domain=-1:4,cyan,samples=50]{0.5*x^2};
	\addplot[domain=-1:4,blue,]{x};
	\end{axis}
\end{tikzpicture}
\par The secant line $y=x$ approximates the line tangent to the graph $f(x)=\frac{1}{2} x^2$ for the point $x=1$.
\end{center}
\paragraph{The Slope of a Secant Line} The slope of a secant line through a point $x$ and another point $h$ units away will be the rise ($\triangle y$) divided by the run ($\triangle x$). Plugging in for the points $(x,f(x))$ and $(x+h,f(x+h))$, we get $$\frac{f(x+h)-f(x)}{x+h-x}$$which solves to$$\frac{f(x+h)-f(x)}{h}$$
\subsection{The Derivative}
\par The derivative is a tool that can find the slope of a tangent line at (almost) any point on a graph. While we can't find the slope between one point and itself because there is no difference, we can find the slope between some point and another that is \textbf{really} close to that point. We can minimize the difference between these two points by using a limit! We will find the difference between a point and a point whose position is almost zero units away: $$\lim_{h\to 0} \frac{f(x+h)-f(x)}{h}$$ This is the derivative! If you were to plug in 0 for $h$ immediately, you would divide by zero. Therefore, you must use the reduction technique for solving equations of this type. The derivative is often abbreviated $\displaystyle\frac{d}{dt} f(t)$ with respect to some variable $t$ and some function $f$; or simply $f'(x)$. If you take $n$ more derivatives of the same function past the first one, we call it the $n^th$ order derivative. Derivatives beyond the first order are labeled $\displaystyle\frac{d^2}{dt^2} f(t)$ or $f''(t)$ for the second order; $\displaystyle\frac{d^3}{dt^3} f(t)$ or $f'''(t)$ for the third order; etc. We say the derivative is taken ``with respect to'' something else. In a pretty standard Calculus case, the objective is to find the change in  \textit{y with respect to x}. We use the notation to describe this relationship as: $\displaystyle\frac{dy}{dx}$. Extending this, the second order derivative of \textit{y with respect to x} is denoted $\displaystyle\frac{d^2y}{dx^2}$.
\paragraph {Example} For $f(x)=x^2$, the derivative is:
\begin{equation}
\begin{split}
&= \lim_{h\to 0} \frac{(x+h)^2-x^2}{h} \\
&= \lim_{h\to 0} \frac{x^2+2xh+h^2-x^2}{h} \\
&= \lim_{h\to 0} \frac{2xh+h^2}{h} \\
&= \lim_{h\to 0} 2x+h \\
&= 2x+0 \\
&= 2x
\end{split}
\end{equation}
So for any point $x$ on the graph of $f(x)=x^2$, the slope of the tangent line will be $y=2x$.
\subsection{Forms}
\par The derivative has patterns, or ``forms'' as they will be called here. You are expected to know these forms backwards and forwards. Many textbooks include proofs for these but, as it is unnecessary, this text will not. The reader is strongly encouraged to prove these forms as exercises. Some of the most important forms that Calculus students are expected to know are below. The derivative of:
\begin{enumerate}
    \item a constant is 0
    \item $u^n$ is $(nu^n-1)\text{ } du$
    \item $cu$ is $c*(u')$
    \item $e^u$ is $e^u\text{ } du$
    \item ln($u$) is $\frac{1}{u}\text{ }du$
    \item sin($u$) is cos($u$) $du$
    \item cos($u$) is -sin($u$) $du$
\end{enumerate}
Where $u$ is some equation and $du$ is the derivative of $u$. Most other forms can be derived from these.
\subsection{Special Rules}
\par There are some more patterns outside of the traditional forms that the reader should know for Calculus. These have to do with composite and combined equations.
\paragraph{The Product Rule} Given two equations $f(x)$ and $g(x)$, the derivative of the function $h(x)=f(x)*g(x)$ is: $$h'(x)=f'(x)g(x)+g'(x)f(x)$$ For example, the derivative of $h(x)=x^2sin(x)$ is: 
\begin{equation}
\begin{split}
h(x)&=f(x)*g(x) \\
h'(x)&=f'(x)g(x)+g'(x)f(x) \\
h'(x) &= 2x*\text{sin}(x)+\text{cos}(x)*x^2
\end{split}
\end{equation}
\paragraph{The Quotient Rule} \textit{Avoid this at all costs.} The quotient rule is helpful but it is very, needlessly complex. For some function $h(x)$ that is the quotient of two functions $f(x)$ and $g(x)$: $h(x)=\displaystyle\frac{f(x)}{g(x)}$$$h'(x)=\frac{g(x)f'(x)-f(x)g'(x)}{(f(x))^2}$$ An easy way to remember this god-awful formula is a little song to the tune of \textit{Low Rider} by War:
\begin{displayquote}
	Low...d...High...minus High d Low...all...o-ver...the square of what's below 
\end{displayquote}
\begin{displayquote}
	(All...my...friends...know the low rider...the...low...rid-er...is a little higher)
\end{displayquote}
All jokes aside, this is one of the worst things in Calculus. Avoid it at all costs. For example, you could rewrite $\displaystyle\frac{3-x}{x^2}$ as $3x^{-2}-x^{-1}$, thereby eliminating all need for the quotient rule.
\paragraph{The Chain Rule} The name is confusing. No, this rule doesn't have anything to do chains in the normal sense. Instead, it gives a form for a composite function $h(x)=f(g(x))$. This is required for functions like sin($x^2$) where sin$(x)=f(x)$ and $x^2=g(x)$. The chain rule says the derivative of $h(x)=f(g(x))$ is:$$h'(x)=f'(g(x))*g'(x)$$In our previous example, the derivative of sin$(x^2)$ is
\begin{equation}
\begin{split}
h(x) &= f(g(x)) \\
h'(x) &= f'(g(x))*g'(x) \\
h'(x) &= \text{cos}(x^2)*2x \\
h'(x) &= 2x*\text{cos}(x^2)
\end{split}
\end{equation}
\subsection{Related Rates}
\par The derivative finds use in many physics and optimization problems. Because it returns the slope, it finds how one thing changes with respect to another. In physics, if you have a position equation $x(t)$, the derivative will return another equation giving the rate of change of position. This has a name in physics: the velocity. Similarly, the derivative of velocity will return the rate of change of velocity: acceleration. These can be summarized in the equations:$$x'(t)=v(t)$$$$x''(t)=a(t)$$
\subsection{Local maxima and minima} For any graph, the local minima and maxima will occur when the graph levels off. To put this in Calculus words, the derivative at the point must be 0 because the tangent line is horizontal.
\begin{center}
	\begin{tikzpicture}
	\begin{axis}[restrict y to domain=-4:4,]
	\addplot[domain=-2:2,purple,samples=50]{3*x*(x-1)*(x+1)};
	\end{axis}
	\end{tikzpicture}
	\newline This graph has local maxima and minima where the tangent line is horizontal.
\end{center} 
\newpage Armed with the knowledge of the derivative, we can exactly calculate at what point(s) any function has a local maximum or minimum. It is actually very simple, all one has to do is to find at what points the derivative is 0. We call this \textit{The First Derivative Test}. The first derivative test gives us \textit{critical numbers}. For example, let's find the points at which $f(x)=\displaystyle\frac{x^3}{3}-x$ has a local maximum or mimimum:
\begin{equation}
\begin{split}
f'(x)&=3\frac{x^2}{3}-x^0 \\
f'(x)&=x^2-1 \\
0 &= x^2-1 \\
0 &= (x-1)(x+1) \\
x&=\pm 1
\end{split}
\end{equation}
\paragraph{I lied.} When the derivative is zero, all it means is the line tangent to the graph is horizontal. For example, the graph of $f(x)=x^3$ has a horizontal tangent when $x=0$:
\begin{center}
	\begin{tikzpicture}
	\begin{axis}[restrict y to domain=-8:8,]
	\addplot[domain=-2:2,green,samples=50]{x^3};
	\end{axis}
	\end{tikzpicture}
\end{center} 
There must be some way to verify if the critical numbers we get from the first derivative test are truly maxima and minima, right? Yes! If we look at the graph of $f(x)=x^3$ above, we can tell that the tangent line has a positive slope moving towards $x=0$ and also moving away. In calculus-speak: the first order derivative does not change signs from positive to negative. If we look at the graph of $f(x)=x^2$, we can see that the slope of the tangent is negative moving towards $x=0$ and positive moving away. This means the first order derivative changes signs. This makes logical sense because for a function to have a local mimimum, it must decrease, hit the lowest point, and increase. We can flip this around, too: for a function to have a local maximum, it must increase, hit the highest point, and come down again. The first order derivative (the slope of the tangent line) must change signs. If it changes from positive to negative, there is a local maximum. If it changes from negative to positive, there is a local mimimum. We can write up this in a table where we have the intervals from the beginning of the values we are checking to the first critical number, then from the first to second critical number, second to third critical number,$\cdots$, and from the last critical number to the end of the values we are testing. ``Values we are testing'' means the ends of the interval we are working with. Normally the ends are $-\infty$ and $\infty$, but sometimes you might be constrained to [0,5] or something similar. For example, to find the relative minima and maxima of the function whose derivative is $f'(x)=(x-1)(x+1)$, this is the table one would create:
\begin{center}
	\begin{tabular}{ |c|c|c|c|c| } 
		\hline
		(-$\infty$, -1) &-1 & (-1, 1) &1 &(1, $\infty$) \\ \hline
		+ & 0 & - &0 &+ \\
		\hline
	\end{tabular}
\end{center}
...and this is the true first derivative test. Because it changes signs from positive to negative at $x=-1$, that point is a local maximum, and because it changes from negative to positive at $x=1$, that must be a local minimum.
\end{document}